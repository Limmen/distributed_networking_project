%% Report template.
%% Author: Kim Hammar
\documentclass[a4paper, 11pt]{article}
\usepackage{graphicx}
%\usepackage[T1]{fontenc}
\usepackage[utf8]{inputenc} 
%\usepackage[swedish]{babel}
\usepackage{graphicx}
\usepackage{cite}
\usepackage{natbib}
\usepackage{url}
\usepackage{float}
\usepackage{vhistory}
\usepackage{lastpage}
\usepackage{fancyhdr}
\usepackage{lettrine}
\graphicspath{ {./images/} }
\usepackage{hyperref}
\hypersetup{
    colorlinks,
    citecolor=black,
    filecolor=black,
    linkcolor=black,
    urlcolor=black
}
\pagestyle{fancy}
\fancypagestyle{firststyle}
{
  \lhead{}
  \chead{}
  \rhead{\textbf{Course project} ID2212}
   \fancyfoot[C]{\footnotesize Page \thepage\ of \pageref{LastPage} }
}
  \lhead{}
  \chead{}
  \rhead{\leftmark}
  \fancyfoot[C]{\footnotesize Page \thepage\ of \pageref{LastPage} }

\title{Course project}
\author{Kim Hammar}
\date{7 dec 2015}
\begin{document}
\maketitle
\thispagestyle{firststyle}
\begin{versionhistory}
\vhEntry{0.1}{07.12.15}{KH}{First draft}
\end{versionhistory}
\newpage
\tableofcontents
\newpage

%\begin{figure}[H]
%\includegraphics[scale=0.5]{klass.png}
%\caption{Konceptuell modell över systemet}
%\end{figure}
\section{Abstract}
Course project in a course on Network programming in Java \citep{kth_1}, carried out at Royal Institute of Technology, Stockholm.
\newpage
\section{Task specification}
\textit{You are ``hired'' by JEM inc (Java Enterprise Microsystems Inc.) to design and develop the distributed application software (clients and servers) for the NOG (Nordic Olympic Games) event.} \\ \\
The NOG information system should allow storing, retrieving and updating personal information about NOG participants. The system should also be able to provide statistical information about participants. The system is to be developed in two version(1) a single-user version; (2) a multi-user version. \\ 
You should also develop a NOG virtual meeting place. The NOG virtual meeting place is Internet based software which offers remotely located users to communicate and share information represented as textual, image or audio files. 
\subsection{Sub-assignment 1. A Single-User Information System for NOG}
Develop a distributed application in Java that allows storing, retrieving and updating information about participants of NOG. The application should concist of a client with a user interface and a server. In this assignment we assume a \textbf{single user} semantics for the application, i.e It's not required to support coherency of multiple copies of participant records which may be cached by multiple client at the same time.
\subsection{Sub-assignment 2. A Multi-User Information System for NOG}
Develop a multi-user application, similar to the solution developed in sub-assignment 1. In this version a multi-user semantic is required. Many users can fetch the participants-data at the same time and when one user updates his local-cache of the data, the change need to be replicated among all clients connected in order to prevent them from using stale data.
\subsection{Sub-assignment 3. Chat Rooms for NOG}
Develop a distributed ``building of chat rooms''.

\newpage
\section{Platform}
The platform used for the devlopment-process, benchmarks and tests is a computer running Xubuntu 14.04 LTS, CPU: Intel(R) Core(TM) i7-4790 CPU @ 3.60GHz \\ \\
Java version: 1.7.0\_79  (OpenJDK version 7 update 79).
\section{Software and technologies used}
\begin{itemize}
\item Java Remote Method Invocation (java.* package)
\item Java Persistence API (java.* package)
\item JDBC (java.* package)
\item Java Swing (java.* package)
\item Java Socket (java.net package)
\item PostgreSQL 9.3.9
\end{itemize}
\newpage
\section{The application}
\subsection{Functionality}
\subsection{Protocols used}
\subsection{GUI}
\subsection{Architecture}

\section{Load-testing}
\subsection{Sub-assignment 1. A Single-User Information System for NOG}
These tests have been done on my local machine so It is'nt a real proof of how the application would hold under huge loads in production but we can still see some interesting results.
My main purpose with this load test was to see how well the multi-threaded semantics is working in reality. Since the test-environment is on a machine with 7 cores, we can expect that the througput would be higher when there is multiple clients sending requests concurrently.
\subsubsection{Without latency simulation}
%%Table with borders 
\begin{table}[H]
\centering
\label{Load-test for http-server}
\begin{tabular}{|l|l|l|r|}
\hline
\textbf{No. threads}  & \textbf{No. requests}  & \textbf{Throughput/sec} & \textbf{(KB/sec)} \\ \hline
1Thread & 100 & 990.099 & 17579.091 \\ \hline
2Threads & 100 & 1470.588 & 26110.122 \\ \hline
4Threads & 100 & 1886.792 & 33499.779 \\ \hline
10Threads & 100 & 2631.579 & 46723.376 \\ \hline
TOTAL & 400 & 1486.989 & 26401.313 \\ \hline
\end{tabular}
\caption{Load-test for http-server}
\end{table}
\begin{figure}[H]
\includegraphics[scale=0.5]{no_delay.png}
\caption{Throughput/sec for different number of threads}
\end{figure}
The result show that 10 threads gave more than doubled the througput compared to 1 thread. But it was'nt the result I expected, I expected a linear growth in throughput with respect to number of threads, until we reach $\approx 7$ threads (which is the maximum number of threads that can run in parallell on the test-machine) where the througput would stabilize around some value.
\subsubsection{With latency simulation}
Since the server was running on my local machine it barely was'ny any latency between the requests at all, I figured that was the reason the tests did'nt match my expectations.
To simulate network-latency that might occur outside of the test-environment I added a 200 millisecond delay at the server while handling the requests and re-did the tests to see if it had any effect.
\\
%%Table with borders 
\begin{table}[H]
\centering
\label{Load-test for http-server with latency simulation}
\begin{tabular}{|l|l|l|r|}
\hline
\textbf{No. threads}  & \textbf{No. requests}  & \textbf{Throughput/sec} & \textbf{(KB/sec)} \\ \hline
1Thread & 100 & 4.946 & 87.817 \\ \hline
2Threads & 100 & 9.870 & 175.253 \\ \hline
4Threads & 100 & 19.685 & 349.505 \\ \hline
10Threads & 100 & 49.189 & 873.334 \\ \hline
TOTAL & 400 & 10.672 & 189.486 \\ \hline
\end{tabular}
\caption{Load-test for http-server with latency simulation}
\end{table}
\begin{figure}[H]
\includegraphics[scale=0.5]{delay.png}
\caption{Throughput/sec for different number of threads}
\end{figure}

As can be seen from the benchmark-results with the latency-simulation, the througput is much lower, obviously. What's interesting here is the througput gains we get by having multiple threads issuing the requests in parallell.
\subsection{Sub-assignment 2. A Multi-User Information System for NOG}
The loadtesting done for this assigment is of another nature than the tests done for sub-assignment 1. Here i have created a custom LoadTesting class for issuing remote method-calls to the RMI-server all method-calls is done single-threaded. An important note here is that all of these functions contains database interaction, so besides the Java RMI server these tests also depend on the database-layer which is in PostgreSQL. Just like for the load-testing done on the sub-assignment 1 http-server, number of calls done for each method is 100.\\
%%Table with borders 
\begin{table}[H]
\centering
\label{Performance test for rmi-server}
\begin{tabular}{|l|l|r|}
\hline
\textbf{Method}  & \textbf{No. calls} & \textbf{Time (s)} \\ \hline
getParticipants	& 100 & 0.926 \\ \hline
addParticipant	& 100 & 0.293 \\ \hline
deleteParticipant & 100 & 0.110 \\ \hline
deleteParticipant & 100 & 0.211 \\ \hline
\end{tabular}
\caption{Performance-test for rmi-server}
\end{table}
\begin{figure}[H]
\includegraphics[scale=0.5]{rmiplot.png}
\caption{Benchmark results.}
\end{figure}
The result is not very suprising, \textit{getParticipant} is by far the method that takes the most  time and \textit{deleteParticipant} takes the least. \\

This was not a sophisticated load-test but we can still see that the rmi-server is alot slower than the http-server in sub-assignment 1, if we convert the data in the table above we can get that the rmi-server can handle 100 calls for getParticipants in $\approx 0.92$ seconds. In comparison with the load-test result from sub-assignment 1 (the single-threaded version) which could handle $\approx 990$ GET-requestst of the participants, the rmi-server is way slower. The fact that the rmi-server is slower than the http-server is not suprising since the http-server simply reads from a tsv file while the rmi-server goes through many more steps: conversion from relational data to object-data with the ORM, compile psql-commands down to sql etc. but I did'nt expect the differencies to be this big.
\section{Documentation}
\newpage
\nocite{*}
\bibliographystyle{plain}
\bibliography{references}
\end{document}


%%% Local Variables:
%%% mode: latex
%%% TeX-master: t
%%% End:

%\textbf // bold
%\textit // italic

%%Punktlista
%%
%%\begin{itemize}
%%  \item The first item
%%  \item The second item
%%  \item The third etc \ldots
%%\end{itemize}

%% Numrerad lista
%%
%%\begin{enumerate}
%%  \item The first item
%%  \item The second item
%%  \item The third etc \ldots
%%\end{enumerate}

%%table without borders:
%%\begin{table}[]
%%\centering
%%\label{my-label}
%%\begin{tabular}{llr}
%%Animal     & Description & Price (\$) \\ \hline
%%Gnat       & per gram    & 13.65      \\
%%           & each        & 0.01       \\
%%Gnu        & stuffed     & 92.50      \\
%%Emu        & stuffed     & 33.33      \\
%%Armadillo  & frozen      & 8.99       \\ \hline
%%\end{tabular}
%%\caption{My caption}
%%\end{table}

%%Table with borders
%% 
%%\begin{table}[]
%%\centering
%%\label{my-label}
%%\begin{tabular}{|l|l|r|}
%%\hline
%%\textbf{Animal}      & \textbf{Description}  & \textbf{Price} \textit{(\$)} \\ \hline
%%Gnat        & per gram     & 13.65      \\ \hline
%%            & each         & 0.01       \\ \hline
%%Gnu         & stuffed      & 92.50      \\ \hline
%%Emu         & stuffed      & 33.33      \\ \hline
%%Armadillo   & frozen       & 8.99       \\ \hline
%%\end{tabular}
%\caption{My caption}
%\end{table}


%%C-c C-c //Kompilerar det som behöver kompileras (latex eller bibtex elr whatever, emacs är smart!
%%C-c C-v //Visa pdf
%%C-c = //Visa innehållsförteckningen
%%C-c C-k //Kill the TeX subprocess (tex-kill-job). 
%%C-c C-f // Invoke TeX on the current file (tex-file).
%%C-c C-l // se output av senaste kommandot, error meddelande t.ex.
%%C-c C-c "XeLaTex" //kompilera med XeLaTex

